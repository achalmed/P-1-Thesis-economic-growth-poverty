\section{\large JUSTIFICACIÓN}

\subsection{Justificación teórica}
	
La pobreza es un problema mundial que se ha venido combatiendo a lo largo de los años, cada país en forma individual ha implementado medidas y políticas con el único fin de reducir su nivel de pobreza o extrema pobreza. En caso particular de Perú, a pesar de los índices favorables en la reducción de la pobreza, en determinadas zonas se observa un margen significativo en la desigualdad socioeconómica el cual es paralelo al índice del nivel pobreza.

Es importante analizar el comportamiento de los índices de pobreza contrastado con indicadores de la desigualdad socioeconómica; para así poder establecer efectividad o importancia a lo largo del tiempo. Crear conocimiento dentro de la relación establecida entre estos indicadores, de manera que se contraste con la teoría económica ya conocida o se generen nuevos aportes.


\subsection{Justificación practica}

Este trabajo tiene una justificación practica porque con él podrá ser posible su aplicación a la realidad para poder hacer frente a la desigualdad socioeconómica que predomina en distintas regiones en particular y sobre todo hacer frente a la pobreza que tanto sufrimiento ocasiona a una gran parte de la población peruana. 

Los resultados de la presente investigación podrán aportar a ver mejor el panorama del país y con él se podrán diseñar estrategias de políticas públicas con los que en un mediano plazo podamos obtener resultados más eficaces en la reducción de la pobreza y la desigualdad socioeconómica en el Perú; además del gran aporte que significara para la toma de decisiones a nivel de gobierno.

\subsection{Justificación metodológica}

Para realizar la investigación sobre la influencia de la desigualdad socioeconómica en la pobreza se empleará el método hipotético deductiva, es decir, iremos de los general a lo particular; primero se hará una visión general de la desigualdad socioeconómica y de la pobreza de la región Latinoamérica para posteriormente, estudiar el caso particular de Perú.










		

