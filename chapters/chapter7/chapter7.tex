\section{\large METODOLOGÍA}

\subsection{Tipo y nivel de investigación}

	\subsubsection{Tipo de investigación}

Por el tipo de investigación, la presente investigación reúne las condiciones metodológicas de una investigación longitudinal según el alcance temporal investigación aplicada, en razón que se utilizaron conocimientos de las ciencias económicas a fin de aplicarlas en desigualad socioeconómica y la pobreza en el departamento Ayacucho, periodo 2000-2019.
	
	\subsubsection{Nivel de investigación}
De acuerdo a la naturaleza del estudio de la investigación reúne por su nivel las características de un estudio descriptivo explicativo y correlacional.

\subsection{Población y muestra}
 
	\subsubsection{Población}
	Según Córdova (2006), la población o universo es la totalidad de personas u objetos que tienen una o más características medibles o contables de naturaleza cualitativa o cuantitativa.\\
Para Hilario (2020), la población es un conjunto de elementos que posee características similares.


	\subsubsection{Muestra}
	
\subsection{Fuentes de información}

Los datos usados son de nivel secundario, es decir, son datos ya procesados. Se ha recurrido a la información brindada por las páginas del INEI y BCRP, pues son las instituciones que han venido desarrollando diferentes encuestas y estudios a lo largo de los años y han consignado datos fidedignos los cuales pueden emplearse en todo tipo de investigación que lo requiera.

\subsection{Diseño de investigación}

Para el diseño de nuestra investigación emplearemos un tipo de investigación no experimental, el cual a su vez es longitudinal porque nuestra variable se estudia para determinar o evolución en el tiempo.

\subsection{Técnicas e instrumentos}

	\subsubsection{Técnicas}
	Las principales técnicas que se utilizaran en esta investigación para el análisis de series de tiempo son:
	\begin{itemize}
	\item Las metodologías econométricas y 
    \item estadística inferencial

    \end{itemize}		
	
	\subsubsection{Instrumentos}
	Los principales instrumentos que se aplicarán en las técnicas son:
	
    \begin{itemize}
    \item Stata
    \item Excel
    \item Eviews

    
    \end{itemize}	



		
