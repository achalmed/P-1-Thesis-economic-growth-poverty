\section{\large VARIABLES E INDICADORES}

\subsection{Desigualdad socio-económica y dimensiones}

\subsubsection{Variable causa}

X= Desigualdad socioeconómica \\
Indicadores:
\begin{itemize}
\item Empleo
\item Índice de Theilen
\item Desempleo
\end{itemize}


\subsubsection{Variable efecto}

Y=Pobreza \\
Indicadores:
\begin{itemize}
\item Índice de Desarrollo Humano
\item Grado de pobreza
\item Índice de Gini
\end{itemize}

	
\subsection{Operacionales de variables y dimensiones}

\begin{table}[H]
  \caption{Matriz de consistencia}
  \label{tab:table1}
\begin{tabular}{llllll}
\hline
\multicolumn{1}{|c|}{Variable}                       & \multicolumn{1}{c|}{Definición conceptual}                                                                                                                                                                                                                                                                                                                                                                                                                  & \multicolumn{1}{c|}{Definición operacional} & \multicolumn{1}{c|}{Indicadores} & \multicolumn{1}{c|}{Unidad de medida} & \multicolumn{1}{c|}{Valor final} \\ \hline
\multicolumn{1}{|l|}{X = Desigualdad socioeconómica} & \multicolumn{1}{l|}{La desigualdad socioeconómica es un problema actual, producto del desarrollo desigual entre las diferentes regiones del mundo y la imposición de ciertas ideologías o definiciones, el precio de unas personas en relación con otras. De hecho, la desigualdad socioeconómica está en la raíz de la discriminación, ya que la desigualdad incluye un trato diferente de quienes se encuentran en desventaja económica, social o moral.} & \multicolumn{1}{l|}{}                       & \multicolumn{1}{l|}{}            & \multicolumn{1}{l|}{}                 & \multicolumn{1}{l|}{}            \\ \hline
\multicolumn{1}{|l|}{Y=Pobreza}                      & \multicolumn{1}{l|}{Según el método de INEI (2007), se define a la pobreza como un grupo de personas que no cumplen con el nivel mínimo de satisfacción para un rango de necesidades básicas relacionadas con la salud, nutrición, educación, vivienda, etc. Es decir, parte de un concepto multidimensional de pobreza considerando diferentes aspectos del desarrollo social.}                                                                            & \multicolumn{1}{l|}{}                       & \multicolumn{1}{l|}{}            & \multicolumn{1}{l|}{}                 & \multicolumn{1}{l|}{}            \\ \hline
                                                     &                                                                                                                                                                                                                                                                                                                                                                                                                                                             &                                             &                                  &                                       &                                 
\end{tabular}
   \begin{tablenotes}[para,flushleft]
     {\small
         \textit{Note.} Elaboración propia.
     }
     \end{tablenotes}
\end{table} 


